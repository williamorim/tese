%% ------------------------------------------------------------------------- %%
\chapter{Polui��o e sa�de p�blica}

\cite{SchwartzDockery1992} conclu�ram que a concentra��o de part�culas suspensas no ar estava positivamente associada com a mortalidade no dia seguinte em Steubenville, Ohio. 

\cite{Saldiva1995} encontraram associa��o positiva entre a mortalidade di�ria em idosos (com idade maior que 65 anos) e a concentra��o de PM10. Os autores n�o conclu�ram n�o existir um n�vel seguro para a concentra��o do poluente no cen�rio estudado.

\cite{Hoek2002} acompanharam uma coorte de 5000 holandeses entre 55 e 59 para investigar a associa��o entre exposi��o a material particulado e morte por doen�as cardiopulmonares. Os autores conclu�ram que o risco de morte estava associado com os n�veis atmosf�ricos do poluente e, mais consistentemente, com viver perto de vias de tr�fego intenso.

\cite{Peters2000} estudaram a chance de interven��es de desfibriladores cardiovasculares implantados em pacientes com hist�rico alto de arritmia. A partir dos resultados de um modelo log�stico, eles conclu�ram que havia associa��o positiva entre o aumento de �xidos de nitrog�nio e o n�mero de arritmias que geravam interven��es.